\chapter{The Graphic User Interface (GUI)}
The Graphic User Interface is in the directory named 
You can run it by typing FreeFem++.tcl\\
\
The Graphic User Interface shows a text window with buttons on the left and right side, a horizontal menu bar above and an entry below where you can see the path of the script when it is opened or where you can type the path of a script to run it.\\
\\
This is the description of the different functions of the GUI :\\
\\
*New : you can access it by the button ``New'' on the right side or by selecting it in the menu File on the horizontal bar.\
It deletes the current script and enables you to type a new script.\\
\\
*Open : you can access it by the button ``Open'' on the right side or by selecting it in the menu File or by typing simultaneously ctrl+o on the keyboard.\
It enables you to select a script which has been saved. This script is then opened in the text window. Then, you can modify it, save the changes, save under another name or run it.\\
\\
*Save : you can access it by the button ``Save'' on the right side or by selecting it in the menu File or by typing simultaneously ctrl+s on the keyboard.\
It enables you to save the changes of a script.\\
\\
*Save as : you can access it by the button ``Save As'' on the right side or by selecting it in the menu File.\\
It enables you to locate where you want to save your script and to choose your the name of your script.\\
\\
*Run : you can access it by the button ``Run'' on the right side or by selecting it in the menu File or by typing simultaneously ctrl+r on the keyboard.\\
It enables you to run the current script.\\
Warning: when you run by typing ctrl+r, the cursor must be in the text window.\\
\\
*Print : you can access it by the button ``Print'' on the right side or by selecting it in the menu File or by typing simultaneously ctrl+p on the keyboard.\\
It enables you to print your script. You have to choose the name of the printer.\\
\\
*Help (under construction) : you can access it by the button ``Help'' on the left side.\\
\\
*Example : you can access it by the button ``Example'' on the left side.\
It runs a little example.\\
\\
*Read Mesh : you can access it by the button ``R.Mesh'' on the left side.\
It enables you to read a mesh which has been saved. It adds the command in your script so that you can use it in your script.\\
\\
*Polygonal Border : you can access it by the button ``P.Border'' on the left side.\\
It enables you to build a polygonal border.\\
When you click on this button, a new window is opened : you have to click on the button ``Border'' then it asks you how many borders you want. When you enter a number and click on ``OK'' the exact number of couple of entries enable you to enter the coordinates of the vertices. And then you have to enter the number of points on each border.\\
When you have finished, you can save by clicking on the button ``S.Mesh''.\\
\\
*Navier Stokes : you can access it by the button ``N.S'' on the left side.\\
A new window is opened. You have to build your polygonal domain by clicking on the button ``Border'' it works like the Polygonal Border function.\\
You can save by clicking on the button ``S.Mesh''. Choose the extension .msh for the name.\\
Then you have to choose the Limits Condition by clicking on the buton ``L.C''.\\
First choose between ``Free'' or ``Imposed', then click on the button ``Validate'' and enter the expression of Imposed condition and then click on ``Validate'' again.\\
\\
*Cut : you can access it by selecting in the menu Edit or by typing simultaneously ctrl+x on the keyboard.\\
\\
*Copy : you can access it by selecting in the menu Edit or by typing simultaneously ctrl+c on the keyboard.\\
\\
*Paste : you can access it by selectiog in the menu Edit or by typing simultaneously ctrl+y on the keyboard.\\

